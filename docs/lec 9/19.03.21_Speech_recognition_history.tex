\input{pre}
\setbeamercolor{alerted text}{fg=blue}
\setbeamersize{text margin left=4mm,text margin right=1mm} 
\setbeamertemplate{frametitle}[default][center]
\setbeamertemplate{navigation symbols}{
	\usebeamerfont{footline}%
    \usebeamercolor[fg]{footline}%
    \hspace{1em}%
    {{\small презентация доступна: \href{https://goo.gl/DUva6P}{\textbf{https://goo.gl/DUva6P}}}
    \hspace{4cm}
    \insertframenumber/\inserttotalframenumber\vspace{0.5mm}}}
\title[]{Computer Speech technologies}
\author[]{G. Moroz}
\date{21 March, 2019}
\begin{document}
\frame{\titlepage}

\section{pre computer}

\begin{frame}{Wolfgang Ritter von Kempelen}
\begin{itemize}
\item \href{https://books.google.ru/books?hl=pl&lr=&id=3MRJAAAAcAAJ&oi=fnd&pg=PA1&dq=Mechanismus+der+menschlichen+Sprache+nebst+Beschreibung+einer+sprechenden+Maschine&ots=MW2q1Xb0dG&sig=qwlXuPxE6P_Ruryv-FXckqV10Pk&redir_esc=y\#v=onepage&q=Mechanismus\%20der\%20menschlichen\%20Sprache\%20nebst\%20Beschreibung\%20einer\%20sprechenden\%20Maschine&f=false}{\cite{kempelen91}}
\end{itemize}
\end{frame}

\section{}
\begin{frame}
{\huge Thank you!\bigskip\\
\normalsize Please, don't hesitate to write me\\
agricolamz@gmail.com
\vspace{-130pt}}
\end{frame}
\begin{frame}{Reference}
\footnotesize
\bibliographystyle{chicago}
\bibliography{bibliography}
\end{frame}
\end{document}
